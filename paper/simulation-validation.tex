\section*{Simulation and Validation}

The simulation of a Petri Net is simple and straightforward. In a same pattern, validation can be done by essentially running interactive simulation on hypothetical cases. These test cases can either be manually created and evaluated or they can be parts of tests suites that have a designated role of evaluating such services. More specifically tests are being fed to the system in order to check if the output is and the expected effects are correct relative with the Petri Net representation.

In the case of verification, composition and performance analysis more advanced techniques are required that also have already been developed in Petri Net. We require techniques that use linear algebra and can be used in order to verify many properties such as place invariants, transitions invariants and possible non-reachability to certain parts of the services. We can also use coverability graph analysis, model checking and common reduction techniques to analyze dynamic behavior of a Petri Net. Finally, in order to do performance analysis and simulation we can use Markov-chain analysis.