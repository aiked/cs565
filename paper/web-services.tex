\section*{Web Services}

\subsection*{Introduction}
A Web service is a method of communication between two electronic devices over a network. It is a software function provided at a network address over the Web with the service always on as in the concept of utility computing. The W3C defines a Web service generally as:-
\begin{quote}
\centering
a software system designed to support inter operable machine-to-machine interaction over a network.
\end{quote}

Web services have automated tools to help generate them. There are approached both bottom-up and top-down oriented. Web services can exist without the necessary presence of the semantic web. Though, combining both the technologies provided by the web services and the semantic web it can give a major advantage and enrich the experience of the users.

\subsection*{Semantic Web Services}
Semantic Web Services, like conventional web services, are the server end of a client–server system for machine-to-machine interaction via the World Wide Web. Semantic services are a component of the semantic web because they use markup which makes data machine-readable in a detailed and sophisticated way (as compared with human-readable HTML which is usually not easily "understood" by computer programs).

The reasons that we were not satisfied with the already existing services is that we have yet problems to address that could be solved using the principles of the semantic web.

The mainstream XML standards for inter operation of web services specify only syntactic interoperability, not the semantic meaning of messages. For example, Web Services Description Language (WSDL) can specify the operations available through a web service and the structure of data sent and received but cannot specify semantic meaning of the data or semantic constraints on the data. This requires programmers to reach specific agreements on the interaction of web services and makes automatic web service composition difficult.

Semantic web services are built around universal standards for the interchange of semantic data, which makes it easy for programmers to combine data from different sources and services without losing meaning.

Currently, Semantic Web services platform use Web Ontology Language(OWL) to allow data and service providers to emantically describe their resources using third-party ontologies like Simple Semantic Web Architecture and Protocol (SSWAP) But the creation and usage of OWL is based on other languages like DARPA Agent Markup Language (DAML) or Ontology Inference Layer (OIL). DAML and OIL were combined and DAML+OIL was created.


