\section*{Semantic Web}

\subsection*{Introduction}
Semantic Web currently is contained in more than four million Webdomains based on a survey done in 2013. But what is exactly the Semantic Web?
The Semantic Web is an extension of the web though standards defined by the World Wide Web Consortium (W3C). These standards promote common data formats and exchange protocols on the Web, most fundamentally the Resource Description Framework (RDF).
According to the W3C, "The Semantic Web provides a common framework that allows data to be shared and reused across application, enterprise, and community boundaries". The term was coined by Tim Berners-Lee for a web of data that can be processed by machines. While its critics have questioned its feasibility, proponents argue that applications in industry, biology and human sciences research have already proven the validity of the original concept.

\subsection*{Standards}
As we said before, the standardization of Semantic Web in the context of Web 3.0 that was defined by Tim Berners-Lee is under the care of W3C. When we talk about standardization we mean the standardization of the formats and the technologies that enable the Semantic Web.
These technologies include:

\begin{itemize}
  \item Resource Description Framework (RDF), a general method for describing information
  \item RDF Schema
  \item Simple Knowledge Organization System (SKOS)
  \item SPARQL, an RDF query language
  \item Notation3 (N3), designed with human-readability in mind
  \item Turtle (Terse RDF Triple Language)
  \item Web Ontology Language (OWL), a family of knowledge representation languages
  \item Rule Interchange Format (RIF), a framework of web rule language dialects supporting rule interchange on the Web
\end{itemize}

\subsection*{Applications}
All these standards enabled us to enhance the usability and the usefulness of the Web and its interconnected resources through.

\begin{itemize}
    \item Servers which expose existing data systems using the RDF and SPARQL standards. Many converters to RDF exist from different applications. Relational databases are an important source. The semantic web server attaches to the existing system without affecting its operation.
    \item Documents "marked up" with semantic information (an extension of the HTML $<$meta$>$ tags used in today's Web pages to supply information for Web search engines using web crawlers). This could be machine-understandable information about the human-understandable content of the document (such as the creator, title, description, etc.) or it could be purely metadata representing a set of facts (such as resources and services elsewhere on the site).
    \item Common metadata vocabularies (ontologies) and maps between vocabularies that allow document creators to know how to mark up their documents so that agents can use the information in the supplied meta-data.
    \item Automated agents to perform tasks for users of the semantic web using this data
    \item Web-based services (often with agents of their own) to supply information specifically to agents, for example, a Trust service that an agent could ask if some online store has a history of poor service or spamming.
\end{itemize}

Take notice of the last bullet because we are going to analyze Web-based services much more and all the challenges its simulation, verification and composition of these services might bring. But before that a quick introduction about what exactly a web service is.