\section*{Analysis of Web Services using Petri Net}
After all this introduction of languages, tools and calculus in order to portray a web services let us visit the subject on how we analyze all these services.
Right now the whole web is full of web services. The type of these web services ranged from web services manually created by provides of web content, by 3rd party web service providers, or by an automated tool.

It is essential that we have a way to perform an analysis on these web services. Features like correctness, effectiveness, safety and efficiency of composite services is very important in order to be able to facilitate a reliable automation in the world of Web services. In this section we provide a variety of computational analysis tools. In order to construct these tools we use languages such as custom made version of DAML-S based on the Petri Net representation.

This enables us the ability to perform the following activities:

\begin{description}
    \item[Simulation] It is essential in order to analyze a web service to test how a Web service can be executed in real life circumstances  under different conditions.
    \item[Validation] You can simulate a web service but is necessary to be able to determine if the running results are as expected. In order to find out if a running service is running as expected it is necessary to perform a validation.
    \item[Verification] During the design and the composition of a service a certain amount of properties are being set. Verification offers us a way to check if the values of these properties match the expected ones. This ensures safety etc along with potentially being an indicator on what exact part you can improve of the web service.
    \item[Composition] A large amount of web services it is possible to generate them using automated tools in order to compose them. The only we have to do is to set a specified goal and a set of properties that has to be matched and respected.
    \item[Performance analysis] Performance is always in a service running in the web. Essentially what we want is to be able to evaluate the ability of a service to meet certain requirements and more specially regarding throughput times, service levels and the level of resource utilization.
\end{description}

All of the above activities must be implemented in tools in order to provide sophisticated automated performance analysis. There are many possible implementation of these models and constructs and we present some of them below that have utilized the DAML-S language we described before.